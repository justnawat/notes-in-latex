\chapter{The Nature of Light and the Principles of Ray Optics}

\section{The Ray Approximation in Ray Optics}

Let us first define the following terms:
\begin{itemize}
    \item \textbf{Wave fronts}: an imaginary \textit{surface} representing points of waves that move 
        in unison
    \item \textbf{Rays}: imaginary lines along the direction that a wave is traveling
    \item \textbf{Geometric optics}: using geometry to analyze how light travels, ignoring inteference
        and diffraction of waves
\end{itemize}

\section{Analysis Model: Wave Under Reflection}

When a light ray hits a surface of another medium, part of that light is reflected. If the surface 
is smooth and every light ray reflected is parallel to each other, we call it \textbf{specular
reflection}. However, if the surface is rough and the reflected light rays aren't so parallel, we
call it \textbf{diffuse reflection}.
\cim{images/ch35/35_5.png}{0.65}

Consider a light ray hitting a smooth surface as shown in the diagram below. The incident and 
reflected angle are $\theta_1$ and $\theta_1'$ respectively. Note that this angle is measured from 
the normal line which perpendicular to the surface.
\cim{images/ch35/35_6.png}{0.35}

The angles can be modeled as 
\begin{equation}\label{35.2}
    \theta_1 = \theta_1'
\end{equation}
as is expressed under the \textbf{law of reflection}.

\section{Analysis Model: Wave Under Refraction}

In addition to the waves being reflected, some of the energy from the incident wave is also transferred
into the new medium. The ray enters the new medium and changes its direction at the boundary. That 
change is called \textbf{refraction}.
\cim{images/ch35/35_10.png}{0.4}

The \textbf{angle of refraction} $\theta_2$ can be modeled with the following relationship$-$Snell's Law.
\begin{equation}\label{35.3}
    \frac{\sin\theta_2}{\sin\theta_1} = \frac{v_2}{v_1} = \frac{n_2}{n_1} = \frac{\lambda_1}{\lambda_2}
\end{equation}
where 
\begin{itemize}
    \item $v_1$ and $v_2$ are the speed of light in the first and second medium respectively
    \item $n = c/v$ where $v$ is speed of light in the medium and $c$ is speed of light in vacuum
        $c \approx 3 \times10^8 m/s$
    \item $\lambda = vf$ as we've established in previous chapters
\end{itemize}

Interestingly, refraction is \textit{reversible}, meaning that refraction would work the same way
for when light is moving from point $A$ in one medium to point $B$ in another medium as when light 
is moving from point $B$ to point $A$.

\subsection{Index of Refraction}

Let us define the \textbf{index of refraction} $n$ of a medium to be the ratio 
\begin{equation}\label{35.4}
    n = \frac{c}{v}
\end{equation}
where $c$ is the speed of light in the vacuum and $v$ is the speed of light in the medium.
This is the same as $n$ from~\eqref{35.3}. Since $v$ is always less than $c$ since light travels 
fastest in the vacuum, $n=1$ for the vacuum. Otherwise, $n > 1$.

Additionally, we can use this to rewrite the equation for Snell's law of refraction as 
\begin{equation}
    n_1\sin\theta_1 = n_2\sin\theta_2
\end{equation}

\section{Total Internal Reflection}

\textbf{Total internal reflection} is an effect that happens when light is travels from a medium 
whose index of refraction is higher than the medium it is traveling to. As a result, refracted rays 
are bent away from the normal line. At some angle of incident $\theta_c$, called the \textbf{critical
angle}, the refracted ray moves parallel to the boundary between the two mediums. This means that 
$\theta_c = 90\degree$. If $\theta_1 > \theta_c$the ray is entirely reflected back into the medium 
it came from.

We can use Snell's Law of Refraction to find $\theta_c$, using $\theta_1 = \theta_c, \theta_2 = 90\degree$.
\begin{align*}
    n_1\sin\theta_c &= n_2\sin\theta_2\\
    n_1\sin\theta_c &= n_2\sin90\degree\\
    \sin\theta_c &= \frac{n_2(1)}{n_1}\\
\end{align*}
Therefore, \begin{equation}
    \sin\theta_c = \frac{n_2}{n_1}
\end{equation}
for some $n_1 > n_2$.
